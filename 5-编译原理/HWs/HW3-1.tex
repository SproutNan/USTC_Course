\documentclass[11pt,letterpaper]{article}
\textwidth 6.5in
\textheight 9.in
\oddsidemargin 0in
\headheight 0in
\usepackage{graphicx}
\usepackage{fancybox}
\usepackage[utf8]{inputenc}
\usepackage[UTF8]{ctex}
\usepackage{epsfig,graphicx}
\usepackage{multicol,pst-plot}
\usepackage{pstricks}
\usepackage{amsmath}
\usepackage{amsfonts}
\usepackage{amssymb}
\usepackage{eucal}
\usepackage[left=2cm,right=2cm,top=2cm,bottom=2cm]{geometry}
\pagestyle{empty}
\DeclareMathOperator{\tr}{Tr}
\newcommand*{\op}[1]{\check{\mathbf#1}}
\newcommand{\bra}[1]{\langle #1 |}
\newcommand{\ket}[1]{| #1 \rangle}
\newcommand{\braket}[2]{\langle #1 | #2 \rangle}
\newcommand{\mean}[1]{\langle #1 \rangle}
\newcommand{\opvec}[1]{\check{\vec #1}}
\renewcommand{\sp}[1]{$${\begin{split}#1\end{split}}$$}

\usepackage{lipsum}

\usepackage{listings}
\usepackage{color}

\definecolor{codegreen}{rgb}{0,0.6,0}
\definecolor{codegray}{rgb}{0.5,0.5,0.5}
\definecolor{codepurple}{rgb}{0.58,0,0.82}
\definecolor{backcolour}{rgb}{0.95,0.95,0.92}

\lstdefinestyle{mystyle}{
	backgroundcolor=\color{backcolour},   
	commentstyle=\color{codegreen},
	keywordstyle=\color{magenta},
	numberstyle=\tiny\color{codegray},
	stringstyle=\color{codepurple},
	basicstyle=\footnotesize,
	breakatwhitespace=false,         
	breaklines=true,                 
	captionpos=b,                    
	keepspaces=true,                 
	numbers=left,                    
	numbersep=5pt,                  
	showspaces=false,                
	showstringspaces=false,
	showtabs=false,                  
	tabsize=2
}

\lstset{style=mystyle}

\begin{document}
\pagestyle{plain}

\begin{flushleft}
姓名:黄瑞轩\\
学号:PB20111686
\end{flushleft}
 
\begin{center}\vspace{-1cm}
\textbf{\large 编译原理作业}\\
HW3-1
\end{center}

 
\rule{\linewidth}{0.1mm}
%%%%%%%%%%%%%%%%%%%%%%%%%%%%%%%%%%%%%%%%%%%%%%%%%%%%%%%%%%%%%%%%%%%%%%%%

\bigskip
\bigskip

\begin{enumerate}

%%%%%%%%%%%%%%%%%%%%
\item Question Number 1
%%%%%%%%%%%%%%%%%%%%

\begin{enumerate}
	\item Escriba un programa en \verb|python| para calcular un valor aproximado de la integral
	\begin{equation}
	\int_0^2\left(x^4 - 2x +1 \right)dx
	\end{equation}
	\item Escriba un programa que calcule la integral usando la regla de Simpson con 10 \textit{slices}.
	\item Compare los resultados con el valor exacto (integre). ¿Cuál es el error fraccional de sus cálculos?
	\item Modifique el programa para utilizar cientos de \textit{slices}, y luego miles. Note la mejora en el resultado. Compare estos resultados con la regla del trapecio utilizando este gran numero de \textit{slices}.
\end{enumerate}

%%%%%%%%%%%%%%%%%%%%
\item Question Number 2
%%%%%%%%%%%%%%%%%%%%

\lipsum[2]

\noindent

\begin{equation}
J_m(x)=\frac{1}{\pi}\int_0^\pi \cos(m\theta-x\sin(\theta))d\theta
\end{equation}

\noindent
\lipsum[1]

Solve

\begin{enumerate}
	\item \lipsum[66]
	\item \lipsum[75]
\end{enumerate}

%%%%%%%%%%%%%%%%%%%%
\item Question Number 3
%%%%%%%%%%%%%%%%%%%%

\lipsum[23]
\begin{itemize}
	\item \lipsum[75]
    \item \lipsum[66]
\end{itemize}

\end{enumerate}


\end{document}

