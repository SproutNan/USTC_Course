% !TeX encoding = UTF-8
% !TeX program = xelatex
% !TeX spellcheck = en_US

%-----------------------------------------------------------------------
% 中国科学: 信息科学 中文模板, 请用 CCT-LaTeX 编译
% http://scis.scichina.com
% 南开大学程明明注释:也可以在Overleaf中使用XeLaTeX直接编译,
% 例如:
%-----------------------------------------------------------------------
\documentclass{SCIS2020cn}
\usepackage{amsthm,amsmath,amssymb,tikz,ctex}
\usepackage{mathrsfs}
\newcommand*{\num}{pi}
 % define the plot style and the axis style
\tikzset{elegant/.style={smooth,thick,samples=50,cyan}}
\tikzset{eaxis/.style={->,>=stealth}}
%\usepackage{breakurl}
%\captionsetup[subfloat]{labelformat=simple,captionskip=0pt}


%%%%%%%%%%%%%%%%%%%%%%%%%%%%%%%%%%%%%%%%%%%%%%%%%%%%%%%
%%% 作者附加的定义
%%% 常用环境已经加载好, 不需要重复加载
%%%%%%%%%%%%%%%%%%%%%%%%%%%%%%%%%%%%%%%%%%%%%%%%%%%%%%%


%%%%%%%%%%%%%%%%%%%%%%%%%%%%%%%%%%%%%%%%%%%%%%%%%%%%%%%
%%% 开始
%%%%%%%%%%%%%%%%%%%%%%%%%%%%%%%%%%%%%%%%%%%%%%%%%%%%%%%
\begin{document}

%%%%%%%%%%%%%%%%%%%%%%%%%%%%%%%%%%%%%%%%%%%%%%%%%%%%%%%
%%% 作者不需要修改此处信息
\ArticleType{电磁学C作业}
%\SpecialTopic{}
%\Luntan{中国科学院学部\quad 科学与技术前沿论坛}
\Year{2020}
\Vol{50}
\No{1}
\BeginPage{-1}
\DOI{}
\ReceiveDate{}
\ReviseDate{}
\AcceptDate{}
\OnlineDate{}
%%%%%%%%%%%%%%%%%%%%%%%%%%%%%%%%%%%%%%%%%%%%%%%%%%%%%%%

\title{黄瑞轩}{黄瑞轩}

\entitle{Title}{Title for citation}

\author[]{黄瑞轩}{}

\enauthor[1]{Mingming XING}{{ }}

\address[1]{ }

\enaddress[1]{Affiliation, City {\rm 000000}, Country}

\AuthorMark{电磁学C作业}

\AuthorCitation{ }
\enAuthorCitation{ }

%\comment{\dag~同等贡献}
%\encomment{\dag~Equal contribution}

\maketitle
\newpage

\part{电力与电场}
\newpage
\part{静电场中的物质与电场能量}
\section{习题2.2}
在未放入导体块之前,电场强度为
\begin{equation}
    E_0=\frac{V}{6L}
\end{equation}

由于间距线度远小于板尺寸,因此之间的电场可视为匀强电场,对左极板,取一柱体形高斯面,由高斯定理得
\begin{equation}
    E_{left}·2S=\frac{Q}{\varepsilon_0}
\end{equation}

故
\begin{equation}
    E_{left}=\frac{Q}{2S\varepsilon_0}
\end{equation}

同理
\begin{equation}
    E_{right}=\frac{Q}{2S\varepsilon_0}
\end{equation}

所以
\begin{equation}
    E_0=E_{left}+E_{right}=\frac{Q}{S\varepsilon_0}=\frac{V}{6L}
\end{equation}

放入导体板后,间距变为$5L$,设导体板左侧离左极板距离$kL$,则右侧离右极板距离$(5-k)L$,此时设左侧电场为$E_1$,右侧电场为$E_2$,对导体板使用高斯定理,有
\begin{equation}
    E_1S-E_2S=\frac{Q}{\varepsilon_0}
\end{equation}

因此
\begin{equation}
    E_1-E_2=\frac{Q}{S\varepsilon_0}=\frac{V}{6L}
\end{equation}

总的电势差满足
\begin{equation}
    V=E_1kL+E_2(5-k)L
\end{equation}

解得
\begin{equation}
    E_1=\frac{V}{5L}\left(1-\frac{k}{6}\right)+\frac{V}{6L},\text{ }E_2=\frac{V}{5L}\left(1-\frac{k}{6}\right)
\end{equation}

由于$E_1,E_2$方向相同,则总的电场力
\begin{equation}
    F=(E_1+E_2)\frac{Q}{2}
\end{equation}

于是需要做功
\begin{equation}
    W=\int_1^3(-F)L\text{d}k=\frac{13\varepsilon_0SV^2}{180L}
\end{equation}

【或者】利用能量关系,移动前两侧电容器的电容分别为
\begin{equation}
    C_{1,left}=\frac{\varepsilon_0S}{L}, C_{1,right}=\frac{\varepsilon_0S}{4L}
\end{equation}

移动后两侧电容器的电容分别为
\begin{equation}
    C_{2,left}=\frac{\varepsilon_0S}{3L}, C_{2,right}=\frac{\varepsilon_0S}{2L}
\end{equation}

相对应的电势
\begin{equation}
    U_{1,left}=E_1·L|_{k=1}
\end{equation}
\begin{equation}
    U_{1,right}=E_2·4L|_{k=1}
\end{equation}
\begin{equation}
    U_{2,left}=E_1·3L|_{k=3}
\end{equation}
\begin{equation}
    U_{2,right}=E_2·2L|_{k=3}
\end{equation}

根据能量守恒
\begin{equation}
    \frac{1}{2}C_{1,left}U_{1,left}^2+\frac{1}{2}C_{1,right}U_{1,right}^2+W=\frac{1}{2}C_{2,left}U_{2,left}^2+\frac{1}{2}C_{2,right}U_{2,right}^2
\end{equation}

即可算出
\begin{equation}
    W=\frac{13\varepsilon_0SV^2}{180L}
\end{equation}
\section{习题2.3}
设各面电荷密度为$\sigma_i$,则有
\begin{equation}
    \sigma_AS+\sigma_BS=5\text{C}
\end{equation}
\begin{equation}
    \sigma_CS+\sigma_DS=1\text{C}
\end{equation}
\begin{equation}
    \sigma_ES+\sigma_FS=1\text{C}
\end{equation}
\begin{equation}
    \sigma_GS+\sigma_HS=2\text{C}
\end{equation}

无限大平板产生的电场强度$E=2\pi{}k\sigma$,则由导体板内部电场强度为0,得到
\begin{equation}
    -2\pi{}k\sigma_A+2\pi{}k\sigma_B+2\pi{}k\sigma_C+2\pi{}k\sigma_D+2\pi{}k\sigma_E+2\pi{}k\sigma_F+2\pi{}k\sigma_G+2\pi{}k\sigma_H=0
\end{equation}
\begin{equation}
    -2\pi{}k\sigma_A-2\pi{}k\sigma_B-2\pi{}k\sigma_C+2\pi{}k\sigma_D+2\pi{}k\sigma_E+2\pi{}k\sigma_F+2\pi{}k\sigma_G+2\pi{}k\sigma_H=0
\end{equation}
\begin{equation}
    -2\pi{}k\sigma_A-2\pi{}k\sigma_B-2\pi{}k\sigma_C-2\pi{}k\sigma_D-2\pi{}k\sigma_E+2\pi{}k\sigma_F+2\pi{}k\sigma_G+2\pi{}k\sigma_H=0
\end{equation}
\begin{equation}
    -2\pi{}k\sigma_A-2\pi{}k\sigma_B-2\pi{}k\sigma_C-2\pi{}k\sigma_D-2\pi{}k\sigma_E-2\pi{}k\sigma_F-2\pi{}k\sigma_G+2\pi{}k\sigma_H=0
\end{equation}

联立上述八式,并利用$Q_i=\sigma_iS$,解得
\begin{equation}
\textbf{\emph{Q}}=\left(\frac{9}{2},\frac{1}{2},-\frac{1}{2},\frac{3}{2},-\frac{3}{2},\frac{5}{2},-\frac{5}{2},\frac{9}{2}\right)\text{C}
\end{equation}

若将$CD$、$EF$两板接通,则有
\begin{equation}
    \sigma_AS+\sigma_BS=5\text{C}
\end{equation}
\begin{equation}
    \sigma_CS+\sigma_DS+\sigma_ES+\sigma_FS=2\text{C}
\end{equation}
\begin{equation}
    \sigma_GS+\sigma_HS=2\text{C}
\end{equation}

$D$、$E$电势相等,其间无电场,故有5个电场强度关系,分别如下:
\begin{equation}
    -2\pi{}k\sigma_A+2\pi{}k\sigma_B+2\pi{}k\sigma_C+2\pi{}k\sigma_D+2\pi{}k\sigma_E+2\pi{}k\sigma_F+2\pi{}k\sigma_G+2\pi{}k\sigma_H=0
\end{equation}
\begin{equation}
    -2\pi{}k\sigma_A-2\pi{}k\sigma_B-2\pi{}k\sigma_C+2\pi{}k\sigma_D+2\pi{}k\sigma_E+2\pi{}k\sigma_F+2\pi{}k\sigma_G+2\pi{}k\sigma_H=0
\end{equation}
\begin{equation}
    -2\pi{}k\sigma_A-2\pi{}k\sigma_B-2\pi{}k\sigma_C-2\pi{}k\sigma_D+2\pi{}k\sigma_E+2\pi{}k\sigma_F+2\pi{}k\sigma_G+2\pi{}k\sigma_H=0
\end{equation}
\begin{equation}
    -2\pi{}k\sigma_A-2\pi{}k\sigma_B-2\pi{}k\sigma_C-2\pi{}k\sigma_D-2\pi{}k\sigma_E+2\pi{}k\sigma_F+2\pi{}k\sigma_G+2\pi{}k\sigma_H=0
\end{equation}
\begin{equation}
    -2\pi{}k\sigma_A-2\pi{}k\sigma_B-2\pi{}k\sigma_C-2\pi{}k\sigma_D-2\pi{}k\sigma_E-2\pi{}k\sigma_F-2\pi{}k\sigma_G+2\pi{}k\sigma_H=0
\end{equation}

联立上述八式,并利用$Q_i=\sigma_iS$,解得
\begin{equation}
\textbf{\emph{Q}}=\left(\frac{9}{2},\frac{1}{2},-\frac{1}{2},0,0,\frac{5}{2},-\frac{5}{2},\frac{9}{2}\right)\text{C}
\end{equation}
\section{习题2.6}
(1)

地球的电势
\begin{equation}
    U_e=\frac{kq_e}{a}
\end{equation}

地球的电容
\begin{equation}
    C_e=\frac{q_e}{U_e}=\frac{a}{k}
\end{equation}

同理,月球的电容
\begin{equation}
    C_m=\frac{q_m}{U_m}=\frac{b}{k}
\end{equation}

月、地组成的电容器为串联关系,有
\begin{equation}
    C=\frac{C_mC_e}{C_m+C_e}=\frac{ab}{k(a+b)}
\end{equation}

(2)

相连后变为并联关系
\begin{equation}
    C=C_m+C_e=\frac{a+b}{k}
\end{equation}
\section{习题2.8}
电容器两极板电量应当相同,因此设内球壳带电$-q_1$,中间壳内壁带电$+q_1$,中间壳外壁带电$-q_2$,外球壳带电$+q_2$,因此电场分布容易写出。

当$r<a$时,
\begin{equation}
    E=0
\end{equation}

当$a<r<b$时,
\begin{equation}
    E=\frac{kq_1}{r^2}
\end{equation}

当$b<r<d$时,
\begin{equation}
    E=\frac{kq_2}{r^2}
\end{equation}

当$d<r$时,
\begin{equation}
    E=0
\end{equation}

以球心为原点,若设$U(\infty)=0$,则由上式,知$U(d)=0$,又因为$U(a)=U(d)$,则$U(a)=0$。则对于$ab$间

\begin{equation}
    U(b)-U(a)=\int_a^bE\text{d}r
\end{equation}

得
\begin{equation}
    U(b)=kq_1\left(\frac{1}{a}-\frac{1}{b}\right)
\end{equation}

因此$ab$间电容为
\begin{equation}
    C_{ab}=\frac{q_1}{U(b)-U(a)}=\frac{ab}{k(b-a)}
\end{equation}

同理$bd$间电容为
\begin{equation}
    C_{bd}=\frac{q_2}{U(d)-U(b)}=\frac{bd}{k(d-b)}
\end{equation}

系统电容为并联关系,则
\begin{equation}
    C=C_{ab}+C_{bd}=\frac{ab}{k(b-a)}+\frac{bd}{k(d-b)}
\end{equation}

若在中间球壳上放入电荷$Q$,则设中间内壁带电$q_1$,外壁带电$q_2$,因此有
\begin{equation}
    \frac{q_1}{C_{ab}}=\frac{q_2}{C_{bd}}
\end{equation}

结合
\begin{equation}
    q_1+q_2=Q
\end{equation}

解得
\begin{equation}
    q_1=\frac{a(d-b)}{b(d-a)}Q, q_2=\frac{d(b-a)}{b(d-a)}Q
\end{equation}

\section{习题2.9}
由于$h<<d$,因此在$b$方向取一微元,可近似认为两板正对。设$\displaystyle\tan\theta=\frac{h}{b}$,则
\begin{equation}
    \text{d}C=\frac{\varepsilon_0a\text{d}x}{d+x\tan\theta}
\end{equation}

整个电容器等于每一微小电容器并联,因此
\begin{equation}
    C=\int_0^b\text{d}C=\frac{ab\varepsilon_0}{h}\ln\left(1+\frac{h}{d}\right)
\end{equation}

\section{习题2.11}
设$C_1$左右电势分别为$\varphi_1,\varphi_3$,左右带电量为$+Q_3,-Q_3$;$C_2$左右电势分别为$\varphi_1,\varphi_2$,左右带电量为$-Q_1,+Q_1$;$C_3$左右电势分别为$\varphi_2,\varphi_3$,左右带电量为$-Q_2,+Q_2$。又观察到题中存在三个孤岛,则本题相当于解方程组
$$\begin{cases}
    Q_1+Q_2+Q_3=C_1V_0\\
    C_1(\varphi_3-\varphi_1)=Q_3\\
    C_2(\varphi_2-\varphi_1)=Q_2\\
    C_3(\varphi_3-\varphi_2)=Q_1\\
    Q_1-Q_2=0\\
    Q_3-Q_1=C_1V_0\\
    -Q_3+Q_2=-C_1V_0\\
\end{cases}$$

其中最后两个方程是等价的,于是六个方程可解六个未知数,得到
$$\begin{cases}
    \displaystyle{}Q_1=Q_2=\frac{C_1C_2C_3V_0}{C_1C_2+C_2C_3+C_3C_1}\\\\
    \displaystyle{}Q_3=\frac{C_1^2(C_2+C_3)V_0}{C_1C_2+C_2C_3+C_3C_1}\\\\
    \displaystyle{}U_1=\varphi_3-\varphi_1=\frac{C_1(C_2+C_3)V_0}{C_1C_2+C_2C_3+C_3C_1}\\\\
    \displaystyle{}U_2=\varphi_2-\varphi_1=\frac{C_1C_3V_0}{C_1C_2+C_2C_3+C_3C_1}\\\\
    \displaystyle{}U_3=\varphi_3-\varphi_2=\frac{C_1C_2V_0}{C_1C_2+C_2C_3+C_3C_1}\\
\end{cases}$$

\section{习题2.13}
(1)

设电极球上的电荷为$+q$,则外壁感应电荷为$-q$,由高斯定理得到两球之间的场强
\begin{equation}
    E=\frac{kq}{r^2}
\end{equation}

则电势差为
\begin{equation}
    U_0=-\int_{R_1}^{R_2}E\text{d}r
\end{equation}

解得
\begin{equation}
    q=\frac{U_0R_1R_2}{k(R_2-R_1)}
\end{equation}

电极处的场强
\begin{equation}
    E_1=\frac{kq}{R_1^2}=\frac{U_0}{R_1(1-\frac{R_1}{R_2})}
\end{equation}

当$R_2\rightarrow\infty$时,该电场最小,最小值为
\begin{equation}
    E_{\min}=\frac{U_0}{R_1}
\end{equation}

(2)

由(1)知道此时
\begin{equation}
    \frac{R_1}{R_2}=\frac{3}{4}
\end{equation}

\section{习题2.15}

(1)

水的密度$\rho=10^3$kg/$\text{m}^3$,则1mol水的体积为
\begin{equation}
    V=\frac{\nu{}M}{\rho}=1.8×10^{-5}\text{m}^3
\end{equation}

单位体积内的分子数为
\begin{equation}
    n=\frac{\nu{}N_A}{V}=3.35×10^{28}\text{m}^{-3}
\end{equation}

由于水分子电矩都朝向同一方向,则极化强度
\begin{equation}
    P=np=0.02\text{C/m}^2
\end{equation}

(2)

体积
\begin{equation}
    V=\frac{4}{3}\pi{}R^3=5.23×10^{-10}\text{m}^3
\end{equation}

分子数
\begin{equation}
    N=\frac{\rho{}VN_A}{M}=1.75×10^{19}
\end{equation}

总的电偶极矩
\begin{equation}
    p_{\text{总}}=Np=1.07×10^{-11}\text{C/m}^2
\end{equation}

由极化的性质知道,外电场的方向与电偶极矩方向一致;由电偶极矩的性质知道,电场强度的大小为
\begin{equation}
    E=\frac{2kp_{\text{总}}}{r^3}=193\text{V/m}
\end{equation}

\section{习题2.17}
由于内部为导体球,故当$r<a$时电场为0。

当$a<r<b$时,由高斯定理
\begin{equation}
    D_1·4\pi{}r^2=q
\end{equation}

故
\begin{equation}
    D_1=\frac{q}{4\pi{}r^2}
\end{equation}

又因为
\begin{equation}
    D_1=\varepsilon{}E_1
\end{equation}

故
\begin{equation}
    E_1=\frac{q}{4\pi\varepsilon{}r^2}
\end{equation}

当$b<r$时,由高斯定理
\begin{equation}
    E_2=\frac{q}{4\pi\varepsilon_0r^2}
\end{equation}

设$\varphi(\infty)=0$,则
\begin{equation}
    \varphi(\infty)-\varphi(b)=-\int_b^{\infty}E_2\text{d}r
\end{equation}

解得
\begin{equation}
    \varphi(b)=\frac{q}{4\pi\varepsilon_0b}
\end{equation}

故当$r>b$时
\begin{equation}
    \varphi(r)=-\int_b^rE_2\text{d}r+\varphi(b)=\frac{q}{4\pi\varepsilon_0r}
\end{equation}

当$a<r<b$时
\begin{equation}
    \varphi(r)=\varphi(b)+\int_r^bE_1\text{d}r=\frac{q}{4\pi\varepsilon_0r}+\frac{q}{4\pi\varepsilon{}r}\left(\frac{1}{r}-\frac{1}{b}\right)
\end{equation}

当$r<a$时
\begin{equation}
    \varphi(r)=\frac{q}{4\pi\varepsilon_0a}+\frac{q}{4\pi\varepsilon{}a}\left(\frac{1}{a}-\frac{1}{b}\right)
\end{equation}

\section{习题2.19}
设墨滴的密度与水的密度相当,为$\rho=10^3$kg/$\text{m}^3$,则加速度为
\begin{equation}
    a=\frac{qU}{md}=\frac{3qU}{4\pi{}r^3\rho{}d}=611\text{m/s}^2
\end{equation}

时间为
\begin{equation}
    t=\frac{L}{u_0}=0.001\text{s}
\end{equation}

故侧向偏移量为
\begin{equation}
    y=\frac{1}{2}at^2=0.31\text{mm}
\end{equation}

速度为
\begin{equation}
    v_y=at=0.6\text{m/s}
\end{equation}

故偏向角度
\begin{equation}
    \theta=\arctan{\frac{v_y}{u_0}}=3.5°
\end{equation}

\section{习题2.21}
(1)

取一个同心球面作为高斯面(半径为$r,a<r<b$),由高斯定理知道
\begin{equation}
    D_1·\frac{1}{2}·4\pi{}r^2+D_2·\frac{1}{2}·4\pi{}r^2=Q
\end{equation}

再一个同心球面作为高斯面(半径为$R,R>b$),由高斯定理知道
\begin{equation}
    E=0
\end{equation}

若取$\varphi(\infty)=0$,则有$\varphi(b)=0$,又因为内球面上电势处处相等,因此
\begin{equation}
    \int_a^bE_1\text{d}r=\int_a^bE_2\text{d}r
\end{equation}

由对称性我们知道$E_1$和$E_2$应当遵守相同的规律,因此存在常数$\lambda$使
\begin{equation}
    E_1=\lambda{}E_2
\end{equation}

联立上述两式得到
\begin{equation}
    E_1=E_2
\end{equation}

解得
\begin{equation}
    E=E_1=E_2=\frac{Q}{2\pi{}r^2(\varepsilon_1+\varepsilon_2)}
\end{equation}

(2)

板间电势
\begin{equation}
    U=\int_a^bE\text{d}r=\frac{Q}{2\pi{}(\varepsilon_1+\varepsilon_2)}\left(\frac{1}{a}-\frac{1}{b}\right)
\end{equation}

因此电容
\begin{equation}
    C=\frac{Q}{U}=\frac{2\pi{}(\varepsilon_1+\varepsilon_2)ab}{b-a}
\end{equation}

\section{习题2.23}
介质球的带电体密度为
\begin{equation}
    \rho=\frac{q_0}{V_0}=\frac{3q_0}{28\pi{}R^3}
\end{equation}

假设金属球上带电为$q$,取一个同心球面作为高斯面(半径为$r,\text{其中}R<r<2R$),由高斯定理知道
\begin{equation}
    D_1·4\pi{}r^2=\rho{}V+q
\end{equation}

得到
\begin{equation}
    D_1=\frac{q_0}{28\pi}\left(\frac{r}{R^3}-\frac{1}{r^2}\right)+\frac{q}{4\pi{}r^2}
\end{equation}

再一个同心球面作为高斯面(半径为$r,r>2R$),由高斯定理知道
\begin{equation}
    D_2=\frac{q_0+q}{4\pi{}r^2}
\end{equation}

因为
\begin{equation}
    D_{1,2}=\varepsilon_0\varepsilon_rE_{1,2}
\end{equation}

因此
\begin{equation}
    \begin{cases}
        \displaystyle{}E_1=\frac{q_0}{28\pi\varepsilon_0\varepsilon_r}\left(\frac{r}{R^3}-\frac{1}{r^2}\right)+\frac{q}{4\pi\varepsilon_0\varepsilon_r{}r^2}\\
        \displaystyle{}E_2=\frac{q_0+q}{4\pi\varepsilon_0\varepsilon_r{}r^2}\\
    \end{cases}
\end{equation}

从无穷远处到金属球表面,电势变化
\begin{equation}
    \Delta{}U=\int_{\infty}^2RE_2\text{d}r+\int_{2R}^RE_1\text{d}r=0
\end{equation}

因此金属球确实带电,其电量解得为
\begin{equation}
    q=-\frac{16}{21}q_0
\end{equation}

结合$\varepsilon_r=2$,金属球表面电势
\begin{equation}
    \varphi(2R)=\int_R^{2R}E_1\text{d}r+\varphi(R)=\frac{5q_0}{168\pi\varepsilon_0R}
\end{equation}
\section{习题2.26}
设该电容器外径为$R_1=5\text{cm}$,内径为$R_2$,电介质的绝对介电常数为$\varepsilon$,所带电荷为$q$。则取同心球为高斯面,由高斯定理有
\begin{equation}
    D·4\pi{}r^2=q
\end{equation}

又$D=\varepsilon{}E$,所以场强
\begin{equation}
    E=\frac{q}{4\pi\varepsilon{}r^2}
\end{equation}

由上式知道,内球壳处的场强为最大,只要此处场强小于等于击穿场强即可,即
\begin{equation}
    E(R_2)=\frac{q}{4\pi\varepsilon{}R_2^2}\leqslant{}E_{\max}
\end{equation}

变形得到
\begin{equation}
    E(r^2)r^2\leqslant{}E(R_2)R_2^2\leqslant{}E_{\max}R_2^2
\end{equation}

则两板之间的电压
\begin{equation}
    U=-\int_{R_2}^{R_1}E\text{d}r=\frac{q}{4\pi\varepsilon}\left(\frac{1}{R_2}-\frac{1}{R_1}\right)=E(r)·r^2\left(\frac{1}{R_2}-\frac{1}{R_1}\right)\leqslant{}E_{\max}\left(R_2-\frac{R_2^2}{R_1}\right)
\end{equation}

当
\begin{equation}
    R_2=\frac{R_1}{2}
\end{equation}
时电势差取得最大值,最大值为
\begin{equation}
    U_{\max}=\frac{R_1E_{\max}}{4}=2.5×10^5\text{V}
\end{equation}

\section{习题2.28}
运用高斯定理可得空间中电场的分布律。

当$r<a$时,电场
\begin{equation}
    \overrightarrow{E}=\frac{qr}{4\pi{}a^3\varepsilon_0\varepsilon_r}\overrightarrow{e_r}
\end{equation}

电位移
\begin{equation}
    \overrightarrow{D}=\frac{qr}{4\pi{}a^3}\overrightarrow{e_r}
\end{equation}

当$r>a$时,电场
\begin{equation}
    \overrightarrow{E}=\frac{q}{4\pi{}r^2\varepsilon_0}\overrightarrow{e_r}
\end{equation}

电位移
\begin{equation}
    \overrightarrow{D}=\frac{q}{4\pi{}r^2}\overrightarrow{e_r}
\end{equation}

储能
\begin{equation}
    W_e=\frac{1}{2}\left(\int_0^a\overrightarrow{D}·\overrightarrow{E}·4\pi{}r^2\text{d}r+\int_a^{\infty}\overrightarrow{D}·\overrightarrow{E}·4\pi{}r^2\text{d}r\right)=\frac{q^2}{8\pi{}a\varepsilon_0}\left(\frac{1}{5\varepsilon_r}+1\right)
\end{equation}
\section{习题2.30}
总的能量
\begin{equation}
    W_e=\frac{1}{2}\left(k\frac{q_1q_2}{r}+k\frac{q_1q_3}{2r}+k\frac{q_2q_1}{r}+k\frac{q_2q_3}{r}+k\frac{q_1q_3}{2r}+k\frac{q_2q_3}{r}\right)=\frac{4kq^2}{r}
\end{equation}

能量转化
\begin{equation}
    W_e=E_{k1}+E_{k2}+E_{k3}
\end{equation}

初始时,以向右为正方向,设三个粒子所受电场力为$F_i(i=1,2,3)$,则受力
\begin{equation}
    F_1=-\left[\frac{kq}{r^2}+\frac{2kq}{(2r)^2}\right]q=-\frac{3kq^2}{2r^2}
\end{equation}
\begin{equation}
    F_2=\left(\frac{kq}{r^2}-\frac{2kq}{r^2}\right)q=-\frac{kq^2}{r^2}
\end{equation}
\begin{equation}
    F_3=\left[\frac{kq}{(2r)^2}+\frac{kq}{r^2}\right]2q=\frac{5kq^2}{2r^2}
\end{equation}

设此时三个粒子的瞬时加速度为$a_i(i=1,2,3)$,则
\begin{equation}
    a_1=\frac{F_1}{m_1}=-\frac{3kq^2}{2mr^2}
\end{equation}
\begin{equation}
    a_2=\frac{F_2}{m_2}=-\frac{kq^2}{2mr^2}
\end{equation}
\begin{equation}
    a_3=\frac{F_3}{m_3}=\frac{kq^2}{2r^2}
\end{equation}

从初始时开始取一个微元$\Delta{}t\rightarrow0$,由上面的形式知道$F$是关于$\Delta{}r$的二阶小量,而位移变化$\Delta{r}$是关于$\Delta{}t$的二阶小量,因此在这一时间微元内可认为外力不变,因此此时位移的变化之比等于加速度之比、速度的变化之比等于加速度之比。设$\Delta{t}$结束时的位置为$r_i'(i=1,2,3)$,速度为$v_i'(i=1,2,3)$,则
\begin{equation}
    r_1-r_1':r_2-r_2':r_3-r_3'=-3:-1:1
\end{equation}
\begin{equation}
    v_1':v_2':v_3'=-3:-1:1
\end{equation}

而
\begin{equation}
    [(r_1-r_1')-(r_2-r_2')]:[(r_2-r_2')-(r_3-r_3')]=1:1
\end{equation}

即$\Delta{t}$结束时,1、2球与2、3球之间的间隔仍然相等。由于此时速度之比仍然等于加速度之比,在下一个$\Delta{t}$时速度变化之比仍将等于加速度之比。因此可知在接下来的每一时刻,三者速度之比都将等于初始时的加速度之比。设三者末速度为$v_i(i=1,2,3)$,则有
\begin{equation}
    v_1:v_2:v_3=-3:-1:1
\end{equation}

因此可得最终三者动能为
\begin{equation}
    \overrightarrow{E_{k}}=(E_{k1},E_{k2},E_{k3})=\left(\frac{9kq^2}{4r},\frac{kq^2}{2r},\frac{5kq^2}{4r}\right)
\end{equation}

\section{习题2.32}
设内球壳的带电量为$+q_1$,则由高斯定理可以写出空间中电场的分布规律。

当$r>R_2$时
\begin{equation}
    E=\frac{q_1+q_2}{4\pi{}\varepsilon_0r^2}
\end{equation}

当$R_1<r<R_2$时
\begin{equation}
    E=\frac{q_1}{4\pi{}\varepsilon_0r^2}
\end{equation}

当$r<R_1$时
\begin{equation}
    E=0
\end{equation}

电势有如下关系
\begin{equation}
    V=\int_{R_1}^{R_2}E\text{d}r+\int_{R_2}^{\infty}E\text{d}r
\end{equation}

解得
\begin{equation}
    q_1=\frac{R_1V}{k}-\frac{R_1}{R_2}q_2
\end{equation}

或者由电势叠加原理得
\begin{equation}
    V=\frac{kq_2}{R_2}+\frac{kq_1}{R_1}
\end{equation}

解得
\begin{equation}
    q_1=\frac{R_1V}{k}-\frac{R_1}{R_2}q_2
\end{equation}

相互作用能
\begin{equation}
    W_{\text{互}}=\frac{1}{2}\left(\frac{kq_2}{R_2}q_1+\frac{kq_1}{R_1}q_2\right)
\end{equation}

自能
\begin{equation}
    W_{\text{自}}=\frac{1}{2}\left(\frac{kq_2}{R_2}q_2+\frac{kq_1}{R_1}q_1\right)
\end{equation}

总的能量
\begin{equation}
    W_e=W_{\text{互}}+W_{\text{自}}=\frac{1}{2}\left[kq_1q_2\left(\frac{1}{R_1}+\frac{1}{R_2}\right)+\frac{kq_2^2}{R_2}+\frac{kq_1^2}{R_1}\right]=\frac{1}{2}\left(q_2V-\frac{R_1}{R_2}Vq_2+\frac{R_1V^2}{k}\right)
\end{equation}

\section{习题2.35}
B板上下面电势相等,设上面带电量为$q_1$,下面带电量为$q_2$。上下两个电容器的电容分别为
\begin{equation}
    C_1=\frac{\varepsilon_0S}{d_1},C_2=\frac{\varepsilon_0S}{d_2}
\end{equation}

因此
\begin{equation}
    U=\frac{q_1}{C_1}=\frac{q_2}{C_2}
\end{equation}

解得
\begin{equation}
    q_1d_1=q_2d_2
\end{equation}

(1)

设该液滴为第$n$滴,则B板总的带电量$Q=q_1+q_2=(n-1)q$,则得
\begin{equation}
    q_1=\frac{(n-1)qd_2}{d_1+d_2}
\end{equation}

上面电场强度
\begin{equation}
    E_1=\frac{U}{d_1}=\frac{q_1}{C_1d_1}=\frac{(n-1)qd_2}{(d_1+d_2)\varepsilon_0S}
\end{equation}

受力平衡
\begin{equation}
    mg=qE_1
\end{equation}

解得
\begin{equation}
    n=\frac{mg\varepsilon_0S(d_1+d_2)}{q^2d_2}+1
\end{equation}

(2)

此时B板带电$Q'=(N-1)q$,上面电场强度为
\begin{equation}
    E_1'=\frac{U'}{d_1}=\frac{q_1'}{C_1d_1}=\frac{(N-1)qd_2}{(d_1+d_2)\varepsilon_0S}
\end{equation}

由能量转化关系
\begin{equation}
    mg(h+H)=qE_1'H
\end{equation}

解得
\begin{equation}
    H=\frac{mgh}{\frac{(N-1)q^2d_2}{(d_1+d_2)\varepsilon_0S}-mg}
\end{equation}

\section{习题2.39}
由题可知,从一侧环的中心点到这一侧环的无穷远处,电势差满足
\begin{equation}
    qU=\frac{1}{2}mv_0^2
\end{equation}

因此从右侧起始点到右侧环的中央时,速度为最小
\begin{equation}
    \frac{1}{2}mv_{\min}^2=\frac{1}{2}mv_1^2-qU
\end{equation}

即
\begin{equation}
    v_{\min}=\sqrt{v_1^2-v_0^2}
\end{equation}

当粒子处于两环的中间位置时,电势为0,受力为0,速度为最大,由于对称,从1环中央到中间位置与从中间位置到2环的电势差应当相等,根据能量转化关系则有
\begin{equation}
    \frac{1}{2}mv_{\max}^2=\frac{1}{2}mv_1^2+qU
\end{equation}

解得
\begin{equation}
    v_{\min}=\sqrt{v_1^2+v_0^2}
\end{equation}

则
\begin{equation}
    \frac{v_{\max}}{v_{\min}}=\sqrt{\frac{v_1^2+v_0^2}{v_1^2-v_0^2}}
\end{equation}

\section{习题2.41}
(1)

假设左边为正电荷。

发生隧穿前两边的电势为
\begin{equation}
    V_{AB}=\frac{Q}{C}
\end{equation}

储能为
\begin{equation}
    W_1=\frac{1}{2}CV_{AB}^2
\end{equation}

发生隧穿后两边的电势为
\begin{equation}
    V_{AB}'=\frac{Q+e}{C}
\end{equation}

储能为
\begin{equation}
    W_2=\frac{1}{2}CV_{AB}'^2
\end{equation}

由题可知
\begin{equation}
    W_2>W_1
\end{equation}

解得
\begin{equation}
    V_{AB}>-\frac{e}{2C}
\end{equation}

若左边为负电荷,类似可得
\begin{equation}
    V_{AB}<\frac{e}{2C}
\end{equation}

综上有
\begin{equation}
    -\frac{e}{2C}<V_{AB}<\frac{e}{2C}
\end{equation}

(2)

代入(1)中结果可得
\begin{equation}
    C=8.01×10^{-16}\text{F}
\end{equation}

(3)

设单电子岛左侧带电$q_1$,右侧带电$q_2$,则
\begin{equation}
    q_1+q_2=-ne
\end{equation}

两侧电势差为
\begin{equation}
    U_1=\frac{q_1}{C_S}
\end{equation}
\begin{equation}
    U_2=\frac{q_2}{C_D}
\end{equation}

总的能量
\begin{equation}
    W_e=\frac{1}{2}C_SU_1^2+\frac{1}{2}C_DU_2^2
\end{equation}

两侧电势差之间有关系
\begin{equation}
    U_1+U_2=V
\end{equation}

因此
\begin{equation}
    W_e=\frac{1}{2}(C_S^{-1}+C_D^{-1})^{-1}V^2+\frac{(-ne)^2}{2(C_S+C_D)}
\end{equation}

由于$V$是常量,因此单电子岛上的静电能为上式第二项,即
\begin{equation}
    W_{e_{island}}=\frac{(-ne)^2}{2(C_S+C_D)}
\end{equation}
\end{document}