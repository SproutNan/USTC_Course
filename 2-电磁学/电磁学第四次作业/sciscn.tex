% !TeX encoding = UTF-8
% !TeX program = xelatex
% !TeX spellcheck = en_US

%-----------------------------------------------------------------------
% 中国科学: 信息科学 中文模板, 请用 CCT-LaTeX 编译
% http://scis.scichina.com
% 南开大学程明明注释:也可以在Overleaf中使用XeLaTeX直接编译,
% 例如:
%-----------------------------------------------------------------------
\documentclass{SCIS2020cn}
\usepackage{amsthm,amsmath,amssymb,tikz,ctex}
\usepackage{mathrsfs}
\newcommand*{\num}{pi}
 % define the plot style and the axis style
\tikzset{elegant/.style={smooth,thick,samples=50,cyan}}
\tikzset{eaxis/.style={->,>=stealth}}
%\usepackage{breakurl}
%\captionsetup[subfloat]{labelformat=simple,captionskip=0pt}


%%%%%%%%%%%%%%%%%%%%%%%%%%%%%%%%%%%%%%%%%%%%%%%%%%%%%%%
%%% 作者附加的定义
%%% 常用环境已经加载好, 不需要重复加载
%%%%%%%%%%%%%%%%%%%%%%%%%%%%%%%%%%%%%%%%%%%%%%%%%%%%%%%


%%%%%%%%%%%%%%%%%%%%%%%%%%%%%%%%%%%%%%%%%%%%%%%%%%%%%%%
%%% 开始
%%%%%%%%%%%%%%%%%%%%%%%%%%%%%%%%%%%%%%%%%%%%%%%%%%%%%%%
\begin{document}

%%%%%%%%%%%%%%%%%%%%%%%%%%%%%%%%%%%%%%%%%%%%%%%%%%%%%%%
%%% 作者不需要修改此处信息
\ArticleType{电磁学C作业}
%\SpecialTopic{}
%\Luntan{中国科学院学部\quad 科学与技术前沿论坛}
\Year{2020}
\Vol{50}
\No{1}
\BeginPage{-1}
\DOI{}
\ReceiveDate{}
\ReviseDate{}
\AcceptDate{}
\OnlineDate{}
%%%%%%%%%%%%%%%%%%%%%%%%%%%%%%%%%%%%%%%%%%%%%%%%%%%%%%%

\title{黄瑞轩}{黄瑞轩}

\entitle{Title}{Title for citation}

\author[]{黄瑞轩}{}

\enauthor[1]{Mingming XING}{{ }}

\address[1]{ }

\enaddress[1]{Affiliation, City {\rm 000000}, Country}

\AuthorMark{电磁学C作业}

\AuthorCitation{ }
\enAuthorCitation{ }

%\comment{\dag~同等贡献}
%\encomment{\dag~Equal contribution}

\maketitle
\newpage

\part{电力与电场}
\part{静电场中的物质与电场能量}
\part{电流与电路}
\newpage
\part{磁力与磁场}
\section{习题4.1}
取一个电流元$\text{d}\overrightarrow{l}$,以导线中点为原点$O$,$OP$方向为$x$轴正方向建立坐标系。由BSL定律(由于电流元到$P$点的距离$r$与题设中的$r$容易混淆,这里我们暂且将题设中的$r$改称$a$。)
\begin{equation}
    \text{d}\overrightarrow{B}=\frac{\mu_0}{4\pi}·\frac{I\text{d}\overrightarrow{l}×\overrightarrow{r}}{r^3}
\end{equation}

方向是垂直纸面向里的,大小
\begin{equation}
    \text{d}B=\frac{\mu_0}{4\pi}·\frac{I\text{d}l\sin\theta}{r^2}=\frac{\mu_0}{4\pi}·\frac{Ia\text{d}x}{(x^2+a^2)^{3/2}}
\end{equation}

因此
\begin{equation}
    B=\int\text{d}B=\int_{-\frac{l}{2}}^{\frac{l}{2}}\frac{\mu_0}{4\pi}·\frac{Ia\text{d}x}{(x^2+a^2)^{3/2}}=\frac{\mu_0I}{2\pi{}a}\frac{l}{\sqrt{l^2+4a^2}}
\end{equation}

这里用到了积分公式
\begin{equation}
    \int\frac{1}{(x^2+a^2)^{3/2}}=\frac{x}{a^2(x^2+a^2)^{1/2}}
\end{equation}

当$l>>a$时,(3)式第二项近似为1,即近似为无限长导线时的情况,即
\begin{equation}
    B\approx\frac{\mu_0I}{2\pi{}a}
\end{equation}

\section{习题4.4}
(1)

先来研究一个半径为$r$的圆环中心处的磁场,设电流方向为顺时针,则磁场方向应该垂直纸面向里,大小
\begin{equation}
    B=\int\text{d}B=\frac{\mu_0}{4\pi}\int_0^{2\pi{}r}\frac{I\text{d}l}{r^2}=\frac{\mu_0I}{2r}
\end{equation}

这里要进行单位化,则单位长度内包含的线圈匝数为
\begin{equation}
    n=\frac{N}{b-a}
\end{equation}

则$r$~$r+\text{d}r$内的线圈产生的磁感应强度
\begin{equation}
    B'\text{d}r=nB\text{d}r=\frac{\mu_0IN}{2r(b-a)}\text{d}r
\end{equation}

总的磁感应强度
\begin{equation}
    B_{O}=\int_a^b{}B'\text{d}r=\frac{\mu_0IN}{2r(b-a)}\ln\frac{b}{a}
\end{equation}

(2)

仍然先来研究一个半径为$r$的圆环中心处的磁场,将$\overrightarrow{r}$分解为$\overrightarrow{r_{\text{平行}}}+\overrightarrow{r_{\text{垂直}}}$,由对称性,垂直方向分量引起的磁感应强度将被抵消,因此只需要考虑平行分量,则(由于电流元到$O$点的距离$r$与题设中的$r$容易混淆,这里我们暂且将题设中的$r$改称$s$。)
\begin{equation}
    \text{d}B=\frac{\mu_0}{4\pi}\frac{Ir\text{d}l}{(r^2+s^2)^{3/2}}
\end{equation}

因此
\begin{equation}
    B=\int_0^{2\pi{}r}\text{d}B=\frac{\mu_0Ir^2}{2(r^2+s^2)^{3/2}}
\end{equation}

则$r$~$r+\text{d}r$内的线圈产生的磁感应强度
\begin{equation}
    B'\text{d}r=nB\text{d}r=\frac{\mu_0INr^2}{2(b-a)(r^2+s^2)^{3/2}}\text{d}r
\end{equation}

总的磁感应强度
\begin{equation}
    B_{S}=\int_a^b{}B'\text{d}r=\frac{\mu_0IN}{2(b-a)}\left[\sinh^{-1}(\frac{b}{s})-\sinh^{-1}(\frac{a}{s})-\frac{b}{\sqrt{b^2+s^2}}+\frac{a}{\sqrt{a^2+s^2}}\right]
\end{equation}

这里用到了积分公式
\begin{equation}
    \int\frac{x^2}{(x^2+a^2)^{3/2}}=\sinh^{-1}(\frac{x}{a})-\frac{x}{\sqrt{x^2+a^2}}
\end{equation}

\section{习题4.8}
电流为
\begin{equation}
    I=\frac{e}{\frac{2\pi{}r}{v}}=1.056×10^{-3}\text{A}
\end{equation}

由上题(1)的结果,磁感应强度应为
\begin{equation}
    B=\int\text{d}B=\frac{\mu_0I}{2r}=12.5\text{T}
\end{equation}












\end{document}