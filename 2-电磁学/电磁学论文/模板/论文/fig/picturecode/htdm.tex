% \usepackage{tikz,tikz-3dplot,animate}
%===========================================绘图示例代码==================================================%
%把绘图命令单独定义出来,使得后续方便使用animateinline环境
\newcommand{\drd}[3]{%
\tdplotsetmaincoords{67}{110}%调整坐标系角度
\begin{tikzpicture}[scale=3,tdplot_main_coords,t/.style={font=\scriptsize\itshape}]
    \def\R{.5}%圆周半径,单独定义出来易于修改
    \coordinate(O) at (0,0,0);%定原点,产生参考点
  
    %设置转动坐标系(Euler rotations)
    \tdplotsetrotatedcoords{#1}{#2}{#3}%#1为进动角,#2为章动角,#3为自转角;若没有\tdplotsetrotatedcoordsorigin{point}设置转动坐标系的原点的话则原点默认与主坐标系重合
    \tdplotdrawarc[fill=gray!15,line width=1pt,draw=blue!50!cyan,tdplot_rotated_coords]{(O)}{\R}{0}{360}{}{};%使用tdplot_rotated_coords切换到转动坐标系,并在转动坐标系中画一个圆轨道,圆弧命令:\tdplotdrawarc[coordinate system, draw styles]{(中心点)}{圆弧半径}{起始角}{终止角}{label options}{label}
    % \draw[teal,->,tdplot_rotated_coords] (O)--(.6,0,0)node[above]{$x'$};
    % \draw[teal,->,tdplot_rotated_coords] (O)--(0,.6,0)node[above]{$y'$};
	% \draw[teal,->,tdplot_rotated_coords] (O)--(0,0,.6)node[above]{$z'$};
    
    %画主坐标系,因涉及到颜色覆盖问题而在转动坐标系之后画
    \shade[ball color=teal!30] (O) node[left,t]{定点} circle (.5pt);
    \draw[-latex] (O)--(1.8,0,0) node[right]{$x$(cm)};
    \draw[-latex] (O)--(0,2,0) node[above]{$y$(cm)};
    \draw[-latex] (O)--(0,0,1.5) node[above]{$z$(cm)};
    \tdplotdrawarc[help lines,dashed,thick]{(O)}{\R}{0}{360}{}{};
	
    %将旋转坐标系中的点对应到主坐标系中(点在主坐标系中的各位置分量分别存放到:\tdplotresx,\tdplotresy,\tdplotresz中)
	\tdplottransformrotmain{0.5}{0}{0}%选择动点,此处选动点为旋转坐标系的x轴与轨道的交点(因为容易选,你也可以随意选。)另外注意:\tdplottransformmainrot{x}{y}{z}能将主坐标系中某一点对应到旋转坐标系中,且各分量放在与之同名的\tdplotresx,\tdplotresy,\tdplotresz中
    \shade[tdplot_rotated_coords,ball color=brown] (0.5,0,0) node[above,t]{动点} circle (.5pt);
    \fill (\tdplotresx,\tdplotresy,0) circle (.2pt);
	\draw[-latex,red] (0,0,0) -- (\tdplotresx,\tdplotresy,\tdplotresz)node[below right,t]{$\vec{r}=(\tdplotresx,\tdplotresy,\tdplotresz)$};
	\draw[dashed,red] (0,0,0) -- (\tdplotresx,\tdplotresy,0);
	\draw[dashed,red] (\tdplotresx,\tdplotresy,\tdplotresz) -- (\tdplotresx,\tdplotresy,0);

	%计算指定点与z轴的夹角theta以及与x轴的夹角phi,结果存储在宏\tdplotrestheta和\tdplotresphi中
	\tdplotgetpolarcoords{\tdplotresx}{\tdplotresy}{\tdplotresz}

    % 标记角
    \tdplotdrawarc[->,olive,t]{(O)}{0.1}{0}{\tdplotresphi}{anchor=north}{$\phi=\tdplotresphi^\circ$}
    \tdplotsetthetaplanecoords{\tdplotresphi}%将phi平面转换到theta平面上,旋转角为:phi
    \tdplotdrawarc[tdplot_rotated_coords,->,olive,t]{(O)}{0.2}{0}{\tdplotrestheta}{above}{$\theta=\tdplotrestheta^\circ$}%注意要用tdplot_rotated_coords切换到所谓的theta平面
\end{tikzpicture}}

%-------------进动、章动、自转----------------%
\begin{center}
\begin{animateinline}[loop,autopause,controls,]{5}
    \multiframe{72}{ijdj=0+5,izdj=0+5,izxj=0+5}{%
    % \edef\ijdj{\ijdj}
    % \edef\izdj{\izdj}
    % \edef\izxj{\izxj}
    \drd{\ijdj}{\izdj}{\izxj}}%如果无法正常使用此命令,可考虑将参数先用\edef展开
\end{animateinline}
\captionof{figure}{动点绕定点转动示意图(进动、章动、自转共存)\label{fig:drd}}
\end{center}

%-------------进动----------------%
        % \begin{animateinline}[loop,autopause,controls,]{10}
        %     \multiframe{120}{ijdj=0+3,izdj=10+0,izxj=0+0}{%为了区别自转,先章动10°
        %     % \edef\ijdj{\ijdj}
        %     % \edef\izdj{\izdj}
        %     % \edef\izxj{\izxj}
        %     \drd{\ijdj}{\izdj}{\izxj}}%如果无法正常使用此命令,可考虑将参数先用\edef展开
        % \end{animateinline}

%-------------章动----------------%      
        % \begin{animateinline}[loop,autopause,controls,]{10}
        %     \multiframe{120}{ijdj=0+0,izdj=0+3,izxj=0+0}{%
        %     % \edef\ijdj{\ijdj}
        %     % \edef\izdj{\izdj}
        %     % \edef\izxj{\izxj}
        %     \drd{\ijdj}{\izdj}{\izxj}}%如果无法正常使用此命令,可考虑将参数先用\edef展开
        % \end{animateinline}  

%-------------自转----------------%
        % \begin{animateinline}[loop,autopause,controls,]{10}
        %     \multiframe{120}{ijdj=0+0,izdj=10+0,izxj=0+3}{%为了区别进动,先章动10°
        %     \edef\ijdj{\ijdj}
        %     \edef\izdj{\izdj}
        %     \edef\izxj{\izxj}
        %     \drd{\ijdj}{\izdj}{\izxj}}%如果无法正常使用此命令,可考虑将参数先用\edef展开
        % \end{animateinline}     
%========================================================================================================%