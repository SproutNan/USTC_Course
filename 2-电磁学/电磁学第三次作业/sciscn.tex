% !TeX encoding = UTF-8
% !TeX program = xelatex
% !TeX spellcheck = en_US

%-----------------------------------------------------------------------
% 中国科学: 信息科学 中文模板, 请用 CCT-LaTeX 编译
% http://scis.scichina.com
% 南开大学程明明注释:也可以在Overleaf中使用XeLaTeX直接编译,
% 例如:
%-----------------------------------------------------------------------
\documentclass{SCIS2020cn}
\usepackage{amsthm,amsmath,amssymb,tikz,ctex}
\usepackage{mathrsfs}
\newcommand*{\num}{pi}
 % define the plot style and the axis style
\tikzset{elegant/.style={smooth,thick,samples=50,cyan}}
\tikzset{eaxis/.style={->,>=stealth}}
%\usepackage{breakurl}
%\captionsetup[subfloat]{labelformat=simple,captionskip=0pt}


%%%%%%%%%%%%%%%%%%%%%%%%%%%%%%%%%%%%%%%%%%%%%%%%%%%%%%%
%%% 作者附加的定义
%%% 常用环境已经加载好, 不需要重复加载
%%%%%%%%%%%%%%%%%%%%%%%%%%%%%%%%%%%%%%%%%%%%%%%%%%%%%%%


%%%%%%%%%%%%%%%%%%%%%%%%%%%%%%%%%%%%%%%%%%%%%%%%%%%%%%%
%%% 开始
%%%%%%%%%%%%%%%%%%%%%%%%%%%%%%%%%%%%%%%%%%%%%%%%%%%%%%%
\begin{document}

%%%%%%%%%%%%%%%%%%%%%%%%%%%%%%%%%%%%%%%%%%%%%%%%%%%%%%%
%%% 作者不需要修改此处信息
\ArticleType{电磁学C作业}
%\SpecialTopic{}
%\Luntan{中国科学院学部\quad 科学与技术前沿论坛}
\Year{2020}
\Vol{50}
\No{1}
\BeginPage{-1}
\DOI{}
\ReceiveDate{}
\ReviseDate{}
\AcceptDate{}
\OnlineDate{}
%%%%%%%%%%%%%%%%%%%%%%%%%%%%%%%%%%%%%%%%%%%%%%%%%%%%%%%

\title{黄瑞轩}{黄瑞轩}

\entitle{Title}{Title for citation}

\author[]{黄瑞轩}{}

\enauthor[1]{Mingming XING}{{ }}

\address[1]{ }

\enaddress[1]{Affiliation, City {\rm 000000}, Country}

\AuthorMark{电磁学C作业}

\AuthorCitation{ }
\enAuthorCitation{ }

%\comment{\dag~同等贡献}
%\encomment{\dag~Equal contribution}

\maketitle
\newpage

\part{电力与电场}
\part{静电场中的物质与电场能量}
\newpage
\part{电流与电路}
\section{习题3.1}
(1)

由电流的微观形式
\begin{equation}
    I=neSu
\end{equation}

取一段线元$\Delta{}l$,这段线元的体积为
\begin{equation}
    V=S\Delta{}l
\end{equation}

这段线元的摩尔数为
\begin{equation}
    \nu=\frac{m}{M}=\frac{\rho{}V}{M}=1×10^{5}\Delta{}V(\text{mol})
\end{equation}

这段线元单位体积所蕴含的自由电子数为
\begin{equation}
    n=\frac{3\nu{}N_A}{\Delta{}V}=1.81×10^{29}
\end{equation}

结合题给数据,得到
\begin{equation}
    u=1.72×10^{-7}\text{m/s}
\end{equation}

(2)

由热运动的知识,方均根速率
\begin{equation}
    \sqrt{\bar{v^2}}=\sqrt{\frac{3KT}{m}}=1.17×10^5\text{m/s}
\end{equation}

(3)

由于
\begin{equation}
    \sigma=n\frac{e^2}{2m}\tau=\frac{1}{\rho}
\end{equation}


得
\begin{equation}
    \tau=1.4×10^{-14}\text{s}
\end{equation}

(4)

平均自由程
\begin{equation}
    \lambda=\sqrt{\bar{v^2}}\tau=1.6×10^{-9}\text{m}
\end{equation}

(5)

由(1),电流密度
\begin{equation}
    j=neu
\end{equation}

又由欧姆定律的微分形式
\begin{equation}
    j=\frac{E}{\rho_{\text{电}}}
\end{equation}

联立得电场强度
\begin{equation}
    E=1.40×10^{-4}\text{V/m}
\end{equation}

\section{习题3.4}
根据Drude模型,电阻率
\begin{equation}
    \rho=\frac{2m}{ne^2\tau}=5.56×10^{-8}\text{Ω·m}
\end{equation}

\section{习题3.7}
设通过第$i$只伏特表的电流为$I_{Vi}$,设通过第$i$只安培表的电流为$I_{Ai}$,$1\leqslant{}i\leqslant50$,设伏特表的电阻为$R$,则根据已知条件得
\begin{equation}
    I_{V1}=I_{A1}-I_{A2}
\end{equation}
\begin{equation}
    U_1=I_{V1}R
\end{equation}

并且可得递推关系
\begin{equation}
    I_{A3}=I_{A2}-I_{V2}
\end{equation}
\begin{equation}
    I_{A4}=I_{A2}-I_{V2}-I_{V3}
\end{equation}

一直到

\begin{equation}
    I_{A50}=I_{A2}-I_{V2}-I_{V3}-...-I_{V49}
\end{equation}

由于
\begin{equation}
    I_{A50}=I_{V50}
\end{equation}

因此所求
\begin{equation}
    \sum_{i=1}^{50}{I_{Vi}R}=U_1+\sum_{i=2}^{49}I_{Vi}R+\left(I_{A2}-\sum_{i=2}^{49}I_{Vi}\right)R=304\text{V}
\end{equation}

\section{习题3.11}
设电动势$\mathscr{E}_i$单独存在时,$r_j$所在的支路通过的电流为$I_{ij}$,于是有
\begin{equation}
    I_{11}=\frac{\mathscr{E}_1}{r_1+\left(\frac{1}{r_2}+\frac{1}{r_3}\right)^{-1}}=\frac{13}{15}\text{A}
\end{equation}
\begin{equation}
    I_{12}=-\frac{I_{11}\left(\frac{1}{r_2}+\frac{1}{r_3}\right)^{-1}}{r_2}=-\frac{8}{15}\text{A}
\end{equation}
\begin{equation}
    I_{13}=-\frac{I_{11}\left(\frac{1}{r_2}+\frac{1}{r_3}\right)^{-1}}{r_3}=-\frac{1}{3}\text{A}
\end{equation}

\begin{equation}
    I_{22}=\frac{\mathscr{E}_2}{r_2+\left(\frac{1}{r_1}+\frac{1}{r_3}\right)^{-1}}=\frac{13}{14}\text{A}
\end{equation}
\begin{equation}
    I_{21}=-\frac{I_{22}\left(\frac{1}{r_1}+\frac{1}{r_3}\right)^{-1}}{r_1}=-\frac{4}{7}\text{A}
\end{equation}
\begin{equation}
    I_{23}=-\frac{I_{22}\left(\frac{1}{r_1}+\frac{1}{r_3}\right)^{-1}}{r_3}=-\frac{5}{14}\text{A}
\end{equation}

\begin{equation}
    I_{33}=\frac{\mathscr{E}_3}{r_3+\left(\frac{1}{r_1}+\frac{1}{r_2}\right)^{-1}}=\frac{6}{7}\text{A}
\end{equation}
\begin{equation}
    I_{32}=-\frac{I_{33}\left(\frac{1}{r_1}+\frac{1}{r_2}\right)^{-1}}{r_2}=-\frac{3}{7}\text{A}
\end{equation}
\begin{equation}
    I_{31}=-\frac{I_{33}\left(\frac{1}{r_1}+\frac{1}{r_2}\right)^{-1}}{r_1}=-\frac{3}{7}\text{A}
\end{equation}

根据电流叠加原理,三条路上的电流为
\begin{equation}
    I_j=\sum_{i=1}^3I_{ij}
\end{equation}

解得
\begin{equation}
    I_1=-0.133\text{A,}I_2=-0.033\text{A,}I_3=0.167\text{A}
\end{equation}

端电压
\begin{equation}
    U_i=\mathscr{E}_i-I_ir_i
\end{equation}

解得
\begin{equation}
    U_1=1.267\text{V,}U_2=1.467\text{V,}U_3=1.533\text{V}
\end{equation}

输出功率
\begin{equation}
    P_i=I_iU_i
\end{equation}

解得
\begin{equation}
    P_1=-0.17\text{W,}P_2=-0.05\text{W,}P_3=0.26\text{W}
\end{equation}

\section{习题3.14}

下面考虑$\Delta$型电路。设电流从节点处流入,从节点$i$处流入的电流记为$I_i$,节点$ij$之间的电压记为$U_{ij}$,则
\begin{equation}
    I_1=\frac{U_{31}}{R_{31}}-\frac{U_{12}}{R_{12}}
\end{equation}
\begin{equation}
    I_2=\frac{U_{12}}{R_{12}}-\frac{U_{23}}{R_{23}}
\end{equation}
\begin{equation}
    I_3=\frac{U_{23}}{R_{23}}-\frac{U_{31}}{R_{31}}
\end{equation}

注意:由于这里规定三个点电势依次降低,因此$U_{ij}$是有正负的,这是上面三式子中符号的来源。

下面考虑Y型电路。
\begin{equation}
    U_{12}=I_1R_1-I_2R_2
\end{equation}
\begin{equation}
    U_{23}=I_2R_2-I_3R_3
\end{equation}
\begin{equation}
    U_{31}=I_3R_3-I_1R_1
\end{equation}

联立上面的式子,解得
\begin{equation}
    R_1=\frac{R_{31}R_{12}}{R_{12}+R_{23}+R_{31}}
\end{equation}
\begin{equation}
    R_2=\frac{R_{12}R_{23}}{R_{12}+R_{23}+R_{31}}
\end{equation}
\begin{equation}
    R_3=\frac{R_{23}R_{31}}{R_{12}+R_{23}+R_{31}}
\end{equation}

反解
\begin{equation}
    R_{12}=\frac{R_1R_2+R_2R_3+R_3R_1}{R_3}
\end{equation}
\begin{equation}
    R_{23}=\frac{R_1R_2+R_2R_3+R_3R_1}{R_1}
\end{equation}
\begin{equation}
    R_{31}=\frac{R_1R_2+R_2R_3+R_3R_1}{R_2}
\end{equation}

\section{习题3.17}
(1)

观察可知,中间那个方格的四个顶点均为无交叉电流通过的点,因此可先将网格拆分成两个“$r-L-r$”并联的形式,这里$L$是由三个方格叠成的$L$型结构,这个结构的通电点为$L$的左上、右下两个节点。进一步地,$L$结构中间的节点为无交叉电流通过的点,因此又可将该结构分为$4r$并$r-\Box-r$结构,这里$\Box$的通电节点为两个对角线。综上可以得到等效电阻为
\begin{equation}
    R=\frac{1}{2}\left[\left(\frac{1}{4r}+\frac{1}{2r+\frac{1}{2}·2r}\right)^{-1}+2r\right]=\frac{13}{7}r
\end{equation}

(2)

观察可知,与通电点直接相连的三个节点均为等势点,因此直接与节点相连的三个电阻并联,中间六个电阻并联,这三个结构之间为串联关系,则等效电阻
\begin{equation}
    R=\frac{1}{3}r+\frac{1}{6}r+\frac{1}{3}r=\frac{5}{6}r
\end{equation}

(3)

观察可知,这个球最上面的顶点与最下面的顶点是等势点,将这两点相连,因此从这两个顶点向四周出发的四个电阻两两之间可看作并联关系,即电阻$\displaystyle{}r\rightarrow\frac{1}{2}r$,于是这个电阻可等效成$\boxtimes$的结构,最外边的四边电阻为$r$,中间四个电阻为$0.5r$,因此等效电阻
\begin{equation}
    R=\left(\frac{1}{r}+\frac{1}{0.5r×2}+\frac{1}{2r+\frac{1}{2}r}\right)^{-1}=\frac{5}{12}r
\end{equation}

\section{习题3.23}
设三个支路的电流从上到下分别为$I_1,I_2,I_3$(方向蕴含在下述方程中),则根据基尔霍夫第一方程,有
\begin{equation}
    I_1+I_2=I_3
\end{equation}

根据基尔霍夫第二方程,两个回路分别列出
\begin{equation}
    \mathscr{E}_3-I_1R_3+I_2R_2-I_1R_4=0
\end{equation}
\begin{equation}
    \mathscr{E}_2-I_3R_1-\mathscr{E}_1-I_2R_2=0
\end{equation}

解得
\begin{equation}
    I_1=\frac{2}{7}\text{A}
\end{equation}
\begin{equation}
    I_2=-\frac{8}{7}\text{A}
\end{equation}

即$R_4$上的电压
\begin{equation}
    U_4=I_1R_4=\frac{12}{7}\text{V}
\end{equation}

通过$R_2$的电流
\begin{equation}
    |I_2|=\frac{8}{7}\text{A}
\end{equation}

\section{习题3.28}
假设电流计中无电流通过,则两端等势,则意味着两个电容器两端的电压是相等的。设平均充电电流分别为$I_1,I_2$,则两条路一定同时完成充电和放电,若不然,假设$C_1$先充满电,$C_2$未充满电,则$C_1$也应当给$C_2$充电,矛盾,故得证。因此

\begin{equation}
    Q_{1,2}=I_{1,2}t
\end{equation}

又因为电容器两端电压相等,设为$U$,因此
\begin{equation}
    C_{1,2}=\frac{Q_{1,2}}{U}=\frac{I_{1,2}t}{U}
\end{equation}

又因为
\begin{equation}
    I_{1,2}=\frac{U'}{R_{1,2}}
\end{equation}

则
\begin{equation}
    C_{1,2}=\frac{U't}{UR_{1,2}}
\end{equation}

即可得到
\begin{equation}
    C_1R_1=C_2R_2
\end{equation}

证毕。

\section{习题3.29}
设某一时刻电流为$\displaystyle{}I=\frac{\text{d}Q}{\text{d}t}$,于是整个回路有
\begin{equation}
    \frac{\text{d}Q}{\text{d}t}R+\frac{Q}{C}=\mathscr{E}
\end{equation}

这个微分方程的解是
\begin{equation}
    Q=C\mathscr{E}(1-\text{e}^{-\frac{t}{RC}})
\end{equation}

充电电流
\begin{equation}
    I=\frac{\text{d}Q}{\text{d}t}=\frac{\mathscr{E}}{R}\text{e}^{-\frac{t}{RC}}
\end{equation}

当$t=1$s时

(1)电荷的增加速率
\begin{equation}
    \frac{\text{d}Q}{\text{d}t}|_{t=1\text{s}}=\frac{\mathscr{E}}{R}\text{e}^{-\frac{1}{RC}}=9.55×10^{-7}\text{A}
\end{equation}

(2)储存能量$W_e=\frac{Q^2}{2C}$,因此其变化率
\begin{equation}
    \frac{\text{d}W_e}{\text{d}t}=\frac{Q}{C}\frac{\text{d}Q}{\text{d}t}=\frac{\mathscr{E}^2}{R}(1-\text{e}^{-\frac{t}{RC}})\text{e}^{-\frac{t}{RC}}
\end{equation}

\begin{equation}
    \frac{\text{d}W_e}{\text{d}t}|_{t=1\text{s}}=1.08×10^{-6}\text{W}
\end{equation}

(3)电阻上的热功率$P=I^2R$,因此
\begin{equation}
    P|_{t=1\text{s}}=2.74×10^{-6}\text{W}
\end{equation}

(4)电源的输出功率$P_{out}=P+\frac{\text{d}W_e}{\text{d}t}$,因此
\begin{equation}
    P_{out}|_{t=1\text{s}}=3.82×10^{-6}\text{W}
\end{equation}
\end{document}